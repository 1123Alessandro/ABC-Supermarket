\section{Results and Discussion}

% TODO: RnD
Due to the nature of the experimentation, for each machine learning model that has been implemented there are 5 versions each with different hyperparameters. However for the interpretation and discussion of results, only the best performing model are taken into consideration. 

\subsection{Logistic Regression results}

Due to the polynomial features, there were 465 columns fed into the models, because of this, it's best to instead take note of the most significant coefficients from the model. Looking at table \ref{tab:lr top5 coef}, We can see that the most significant features are all interaction features

\begin{table}[H]
    \caption{Top 5 Features in the logistic regression model}
    \label{tab:lr top5 coef}
    \begin{tabularx}{\linewidth}{l>{\centering\arraybackslash}X}
        \toprule
        Feature & Value \\
        \midrule
        MntFruits \\ A\_Marital\_Status\_Single\_Kidhome & 0.979750 \\
        \midrule
        MntMeatProducts \\ A\_Marital\_Status\_Married\_Teenhome & 0.952025 \\
        \midrule
        MntWines \\ A\_Marital\_Status\_Together\_Teenhome & 0.844306 \\
        \midrule
        Education\_Master \\ A\_Marital\_Status\_Single\_Kidhome & 0.814164 \\
        \midrule
        MntWines \\ NumWebPurchases & 0.763514 \\
        \bottomrule
    \end{tabularx}
\end{table}

It is observable that people who bought fruits, are married, with a kid at home, are very likely to respond and accept the offer. The likeliness to accept the offer can also be seen with married people with a teen at home where they also bought meat products in the last two years, people that live with someone (\texttt{Together} marital status value) that buys wines with a teen at home, single people who achieved masters education with a kid at home, as well as people who bought wines in the last two years that also bought products at the company's website.

\subsection{SVM}

Since the best model tested in the SVC models is a linear model, we can interpret the coefficients to understand how the model has learned. Table \ref{tab:svm top5 coef} shows a lot of similarity with the top five coefficients of the linear regression model. 

\begin{table}[H]
    \caption{Top 5 features in the SVM model}
    \label{tab:svm top5 coef}
    \begin{tabularx}{\linewidth}{l>{\centering\arraybackslash}X}
        \toprule
        Feature & Value \\
        \midrule
        MntWines NumWebPurchases & 0.904773 \\
        A\_Marital\_Status\_Married\_Kidhome \\ A\_Marital\_Status\_Married\_Teenhome & 0.804857 \\
        MntFruits \\ A\_Marital\_Status\_Single\_Kidhome & 0.803837 \\
        NumWebVisitsMonth \\ A\_Marital\_Status\_Single\_Kidhome & 0.769760 \\
        Recency \\ Days\_Since\_Customer & 0.733147 \\
        \bottomrule
    \end{tabularx}
\end{table}

As it turns out, the support vector machine model's top five coefficients has two features similar to the top five of the linear regression model. This would make sense because although they are different machine learning models, grid search algorithm has decided that the best hyperparameters for the SVM included that the svm should use a linear kernel. This would imply that the data is much akin to being linearly separable. Table \ref{tab:svm top5 coef} implies the same ideas as it did for the features that were also in table \ref{tab:lr top5 coef}. For the features unique to table \ref{tab:svm top5 coef}, they simply imply that people who are married with children and teenagers at home, single with a kid at home that visits the company's website in the past month, and people who haven't bought in a while (high \texttt{Recency} value) that are also long standing customers (high \texttt{Days\_Since\_Customer} value) are more willing to respond and accept the offer.

It is a bit difficult to interpret the support vectors found by the SVM. Mainly because any aggregation to these found support vectors are only slightly different to the aggregations found in the original dataset. For example, the average \texttt{Year\_Birth} for the dataset is 1970.325371 while the average for the support vectors is 1970.030303; this is reflected to the other attributes even with different aggregation methods, whether it is the average, standard deviation, etc. the results only vary by a few units. This could be explained by the idea that maybe much of the data points are closer to decision boundary than expected. According to the model there are a total of 198 support vectors found in the dataset; 131 for no responses, and 67 for the responses. To put into perspective, out of the total 971 entries in the training set, 131 are support vectors, that is about 20.3\% of the training set. 

\subsection{Naive Bayes}

According to the implemented Bernoulli Naive Bayes, the log probability of a 0 response (not accepted) is -0.09726884, while a 1 response (accepted the offer) is -2.3785168. This means that the chance of accepting the offer is much lower in probability compared to rejecting it which would make sense given there is a heavy imbalance in the dataset where 1101 responses are 0 while only 113 entries have a response of 1. 

When analyzing the log probability of a feature, the values imply the probability of a feature for a given class, i.e. $P(x_i \mid y)$. Just like the analysis of coefficients at the logistic regression and SVM models, there is too much features to be able to show them all. So, just as before, it is best to just look at the highest scoring (or alternatively, the lowest scoring) of the feature log probabilities. 

\begin{table}[H]
    \begin{tabularx}{\linewidth}{l>{\centering\arraybackslash}X}
        \toprule
        & $ \\
        \midrule
        Year_Birth & -0.624380 \\
        Year_Birth Recency & -0.641402 \\
        NumCatalogPurchases Days_Since_Customer & -0.650023 \\
        NumCatalogPurchases & -0.650023 \\
        Recency & -0.650023 \\
        \bottomrule
    \end{tabularx}
\end{table}

\subsection{Decision Trees}

\subsection{K-Nearest Neighbor}
