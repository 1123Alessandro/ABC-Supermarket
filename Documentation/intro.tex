\section{Introduction}

% TODO: intro

Supermarkets, like any other business, need to meticulously and carefully plan out their business decisions because one wrong move could result in millions in revenue losses. However, risky business decisions are what often create the most capital, popularity, and overall growth for a company, as these decisions create new avenues by attracting more customers or strengthening existing ones. Risky business decisions do not always need to be a gamble; there are many ways to turn them into a calculated risk. This approach allows a company to somewhat have an idea of the possible gains, losses, and approaches to the risks. The most traditional method for calculating risks is by listening to the veterans of the business. However, there are now more effective ways to help business decisions, one being leveraging data. Data and machine learning can help by providing models that predict outcomes to an accurate level that would help in the approach to risky business decisions. 

\subsection{Problem and Dataset}

The lab exercise is provided a dataset that contains various pieces of personal information about customers. Specifically, the dataset contains the columns as shown in table \ref{tab:columns}.

The \texttt{Response} column will be the target for the machine learning models. To interpret, the goal of the machine learning models is to predict which customers accepted the offer in the last campaign. Doing so would give ABC Supermarkets an idea who might be willing to accept the offer of their year-end sale campaign for existing customers. 

\begin{table}[H]
    \caption{Table of the dataset attributes and their data type}
    \label{tab:columns}
    \begin{tabularx}{\linewidth}{l>{\centering\arraybackslash}X}
        \toprule
        Column & Data type \\
        \midrule
        ID & int64\\
        Year\_Birth & int64\\
        Education & object\\
        Marital\_Status & object\\
        Kidhome & int64\\
        Teenhome & int64\\
        Dt\_Customer & object\\
        Recency & int64\\
        MntWines & int64\\
        MntFruits & int64\\
        MntMeatProducts & int64\\
        MntFishProducts & int64\\
        MntSweetProducts & int64\\
        MntGoldProds & int64\\
        NumDealsPurchases & int64\\
        NumWebPurchases & int64\\
        NumCatalogPurchases & int64\\
        NumStorePurchases & int64\\
        NumWebVisitsMonth & int64\\
        Response & int64\\
        Complain & int64\\
        \bottomrule
    \end{tabularx}
\end{table}

